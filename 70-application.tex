\section*{ПРИЛОЖЕНИЕ A}
\addcontentsline{toc}{section}{ПРИЛОЖЕНИЕ A}


\section*{Скрипты создания объектов БД}
В листингах \ref{lst:create_db_1} -- \ref{lst:create_db_2} представлены скрипты создания объектов БД.

\renewcommand{\thelisting}{А.\arabic{listing}}
\setcounter{listing}{0}

\begin{code}
	\captionof{listing}{Скрипт создания объектов БД. Часть 1.}
	\label{lst:create_db_1}
	\inputminted
	[
	frame=single,
	framerule=0.5pt,
	framesep=20pt,
	fontsize=\footnotesize,
	tabsize=4,
	linenos,
	numbersep=5pt,
	xleftmargin=10pt,
	firstline=1,
	lastline=27,
	]
	{text}
	{lst/create-db.sql}
\end{code}

\clearpage


\begin{code}
	\captionsetup{aboveskip=0pt, skip=-5mm}
	\captionof{listing}{Скрипт создания объектов БД. Часть 2.}
	\label{lst:create_db_2}
	\inputminted
	[
	frame=single,
	framerule=0.5pt,
	framesep=20pt,
	fontsize=\footnotesize,
	tabsize=4,
	linenos,
	numbersep=5pt,
	xleftmargin=10pt,
	firstline=29,  
	lastline=59,  
	]
	{text}
	{lst/create-db.sql}
\end{code}

\clearpage

\section*{ПРИЛОЖЕНИЕ Б}
\addcontentsline{toc}{section}{ПРИЛОЖЕНИЕ Б}
\renewcommand{\thelisting}{Б.\arabic{listing}}
\setcounter{listing}{0}

\section*{Скрипты создания ролевой модели БД}

В листингах \ref{lst:role} -- \ref{lst:role_2}  представлен скрипт создания ролевой модели БД.

\begin{code}
	\captionof{listing}{Скрипт создания ролевой модели БД. Часть 1.}
	\label{lst:role}
	\inputminted
	[
	frame=single,
	framerule=0.5pt,
	framesep=20pt,
	fontsize=\footnotesize,
	tabsize=4,
	linenos,
	numbersep=5pt,
	xleftmargin=10pt,
	firstline=1,  
	lastline=24,  
	]
	{text}
	{lst/role.sql}
\end{code}

\clearpage

\begin{code}
	\captionsetup{aboveskip=0pt, skip=-5mm}
	\captionof{listing}{Скрипт создания ролевой модели БД. Часть 2.}
	\label{lst:role_2}
	\inputminted
	[
	frame=single,
	framerule=0.5pt,
	framesep=20pt,
	fontsize=\footnotesize,
	tabsize=4,
	linenos,
	numbersep=5pt,
	xleftmargin=10pt,
	firstline=25,  
	]
	{text}
	{lst/role.sql}
\end{code}

\newpage

\section*{ПРИЛОЖЕНИЕ В}
\addcontentsline{toc}{section}{ПРИЛОЖЕНИЕ В}
\renewcommand{\thelisting}{В.\arabic{listing}}
\setcounter{listing}{0}


\section*{Скрипты создания триггера}
В листинге \ref{lst:trigger} представлен скрипт создания триггера для обновления количества часов, затраченных на проект.

\begin{code}
	\captionof{listing}{Скрипт создания триггера для обновления количества затраченных на проект часов.}
	\label{lst:trigger}
	\inputminted
	[
	frame=single,
	framerule=0.5pt,
	framesep=20pt,
	fontsize=\footnotesize,
	tabsize=4,
	linenos,
	numbersep=5pt,
	xleftmargin=10pt,
	]
	{text}
	{lst/trigger.sql}
\end{code}

\section*{ПРИЛОЖЕНИЕ Г}
\addcontentsline{toc}{section}{ПРИЛОЖЕНИЕ Г}
\renewcommand{\thelisting}{Г.\arabic{listing}}
\setcounter{listing}{0}


\section*{Паттерны взаимодействия с PostgresSQL и Redis}

\begin{code}
	\captionof{listing}{Взаимодействие с PostgreSQL. Часть 1}
	\label{lst:postgres1}
	\inputminted
	[
	frame=single,
	framerule=0.5pt,
	framesep=20pt,
	fontsize=\footnotesize,
	tabsize=4,
	linenos,
	numbersep=5pt,
	xleftmargin=10pt,
	firstline=1,  
	lastline=31,  
	]
	{text}
	{lst/p1.go}
\end{code}
\newpage
\begin{code}
	\captionsetup{aboveskip=0pt, skip=-5mm}
	\captionof{listing}{Взаимодействие с PostgreSQL. Часть 2}
	\label{lst:postgres2}
	\inputminted
	[
	frame=single,
	framerule=0.5pt,
	framesep=20pt,
	fontsize=\footnotesize,
	tabsize=4,
	linenos,
	numbersep=5pt,
	xleftmargin=10pt,
	firstline=32,  
	lastline=67,  
	]
	{text}
	{lst/p1.go}
\end{code}

\newpage

\begin{code}
	\captionsetup{aboveskip=0pt, skip=-5mm}
	\captionof{listing}{Взаимодействие с PostgreSQL. Часть 3}
	\label{lst:postgres3}
	\inputminted
	[
	frame=single,
	framerule=0.5pt,
	framesep=20pt,
	fontsize=\footnotesize,
	tabsize=4,
	linenos,
	numbersep=5pt,
	xleftmargin=10pt,
	firstline=68,  
	lastline=99,  
	]
	{text}
	{lst/p1.go}
\end{code}

\newpage
\begin{code}
		\captionsetup{aboveskip=0pt, skip=-5mm}
	\captionof{listing}{Взаимодействие с PostgreSQL. Часть 4}
	\label{lst:postgres4}
	\inputminted
	[
	frame=single,
	framerule=0.5pt,
	framesep=20pt,
	fontsize=\footnotesize,
	tabsize=4,
	linenos,
	numbersep=5pt,
	xleftmargin=10pt,
	firstline=100,  
	lastline=134,  
	]
	{text}
	{lst/p1.go}
\end{code}

\newpage
\begin{code}
	\captionsetup{aboveskip=0pt, skip=-5mm}
	\captionof{listing}{Взаимодействие с Redis. Часть 1.}
	\label{lst:redis1}
	\inputminted
	[
	frame=single,
	framerule=0.5pt,
	framesep=20pt,
	fontsize=\footnotesize,
	tabsize=4,
	linenos,
	numbersep=5pt,
	xleftmargin=10pt,
	firstline=1,  
	lastline=36,  
	]
	{text}
	{lst/r1.go}
\end{code}
\newpage

\begin{code}
	\captionsetup{aboveskip=0pt, skip=-5mm}
	\captionof{listing}{Взаимодействие с Redis. Часть 2.}
	\label{lst:redis2}
	\inputminted
	[
	frame=single,
	framerule=0.5pt,
	framesep=20pt,
	fontsize=\footnotesize,
	tabsize=4,
	linenos,
	numbersep=5pt,
	xleftmargin=10pt,
	firstline=37,  
	lastline=60,  
	]
	{text}
	{lst/r1.go}
\end{code}


\newpage

\section*{ПРИЛОЖЕНИЕ Д}
\addcontentsline{toc}{section}{ПРИЛОЖЕНИЕ Д}
\renewcommand{\thelisting}{Д.\arabic{listing}}
\setcounter{listing}{0}

\section*{Тестирование}
\begin{code}
	\captionsetup{aboveskip=0pt, skip=-5mm}
	\captionof{listing}{Фикстуры.}
	\label{lst:fixture}
	\inputminted
	[
	frame=single,
	framerule=0.5pt,
	framesep=20pt,
	fontsize=\footnotesize,
	tabsize=4,
	linenos,
	numbersep=5pt,
	xleftmargin=10pt,
	]
	{text}
	{lst/fixture.py}
\end{code}

\begin{code}
	\captionsetup{aboveskip=0pt, skip=-5mm}
	\captionof{listing}{Тестирование на примере сущности "проект "\ . Часть 1.}
	\label{lst:test}
	\inputminted
	[
	frame=single,
	framerule=0.5pt,
	framesep=20pt,
	fontsize=\footnotesize,
	tabsize=4,
	linenos,
	numbersep=5pt,
	xleftmargin=10pt,
	firstline=1,  
	lastline=36,  
	]
	{text}
	{lst/test.py}
\end{code}

\begin{code}
	\captionsetup{aboveskip=0pt, skip=-5mm}
	\captionof{listing}{Тестирование на примере сущности "проект "\ . Часть 2.}
	\label{lst:test_2}
	\inputminted
	[
	frame=single,
	framerule=0.5pt,
	framesep=20pt,
	fontsize=\footnotesize,
	tabsize=4,
	linenos,
	numbersep=5pt,
	xleftmargin=10pt,
	firstline=37,  
	lastline=71,  
	]
	{text}
	{lst/test.py}
\end{code}

\newpage

\begin{code}
	\captionsetup{aboveskip=0pt, skip=-5mm}
	\captionof{listing}{Тестирование на примере сущности "проект "\ . Часть 3.}
	\label{lst:test_3}
	\inputminted
	[
	frame=single,
	framerule=0.5pt,
	framesep=20pt,
	fontsize=\footnotesize,
	tabsize=4,
	linenos,
	numbersep=5pt,
	xleftmargin=10pt,
	firstline=72,  
	lastline=105,  
	]
	{text}
	{lst/test.py}
\end{code}

\newpage

\begin{code}
	\captionsetup{aboveskip=0pt, skip=-5mm}
	\captionof{listing}{Тестирование на примере сущности "проект "\ . Часть 4.}
	\label{lst:test_4}
	\inputminted
	[
	frame=single,
	framerule=0.5pt,
	framesep=20pt,
	fontsize=\footnotesize,
	tabsize=4,
	linenos,
	numbersep=5pt,
	xleftmargin=10pt,
	firstline=106,  
	lastline=140,  
	]
	{text}
	{lst/test.py}
\end{code}

\newpage

\begin{code}
	\captionsetup{aboveskip=0pt, skip=-5mm}
	\captionof{listing}{Тестирование на примере сущности "проект "\ . Часть 5.}
	\label{lst:test_5}
	\inputminted
	[
	frame=single,
	framerule=0.5pt,
	framesep=20pt,
	fontsize=\footnotesize,
	tabsize=4,
	linenos,
	numbersep=5pt,
	xleftmargin=10pt,
	firstline=141,  
	lastline=176,  
	]
	{text}
	{lst/test.py}
\end{code}

\begin{code}
	\captionsetup{aboveskip=0pt, skip=-5mm}
	\captionof{listing}{Тестирование на примере сущности "проект "\ . Часть 6.}
	\label{lst:test_6}
	\inputminted
	[
	frame=single,
	framerule=0.5pt,
	framesep=20pt,
	fontsize=\footnotesize,
	tabsize=4,
	linenos,
	numbersep=5pt,
	xleftmargin=10pt,
	firstline=177,
	]
	{text}
	{lst/test.py}
\end{code}

\pagebreak