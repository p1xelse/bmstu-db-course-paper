\section*{РЕФЕРАТ}

Объектом разработки является информационная система для онлайн трекера времени.

Цель работы -- спроектировать и реализовать базу данных, содержащую данные учёта затраченного времени пользователей, разработать программное обеспечение, которое позволит работать с этой базой данных.

Чтобы достигнуть поставленной цели, требуется решить следующие задачи:
\begin{itemize}[leftmargin=1.6\parindent]
	\item проанализировать варианты модели данных и выбрать подходящий вариант для решения задачи;
	\item проанализировать существующие СУБД и выбрать удовлетворяющие требованиям к хранению данных;
	\item спроектировать базу данных, описать ее сущности и связи;
	\item реализовать программное обеспечение, предоставляющее интерфейс доступа к данным.
\end{itemize}

Готовое приложение позволяет создавать записи о затраченном времени, создавать проекты, цели, а также добавлять пользователей в друзья и смотреть их статистику.

Приложение реализовано на языке Golang, в качестве СУБД была выбрана PostgreSQL.

Результатом исследования является уменьшение времени ответа конечной точки за счет кеширования ответов на 15\%.

КЛЮЧЕВЫЕ СЛОВА: трекер времени, базы данных, SQL, NoSQL, \\PostgreSQL, Redis, Docker, Golang, REST API, кеширование.

Расчетно-пояснительная записка \pageref{LastPage} с., \totalfigures\ рис., \totaltables\ табл., 17 ист., 5 прил.

\pagebreak